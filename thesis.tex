\documentclass[12pt,a4paper]{report}
\usepackage[english, polish]{babel}
\usepackage{polski}
\frenchspacing
\usepackage{indentfirst}
\usepackage{xltxtra}
\usepackage{titlesec}
\usepackage[margin=25mm]{geometry}
\usepackage{nameref}
\usepackage[pdfauthor={Marceli Grabowski}, pdftitle={Wieloplatformowa aplikacja mobilna do zakupu biletów kinowych w technologii Xamarin}]{hyperref}
\usepackage[nottoc,notlot,notlof]{tocbibind}
\usepackage{todonotes}
\usepackage{graphicx}
\usepackage{caption}
\usepackage{subcaption}

\usepackage{float}
\usepackage{pdfpages}
\usepackage{underscore}
\graphicspath{ {img/} }
\newcommand\todoin[2][]{\todo[inline, caption={2do}, #1]{
\begin{minipage}{\textwidth-4pt}#2\end{minipage}}}
\linespread{1}
\titleformat*{\section}{\fontsize{14pt}{2}\bfseries}
\titleformat*{\subsection}{\fontsize{13pt}{2}\bfseries}
\titleformat*{\subsubsection}{\fontsize{13pt}{2}\bfseries}

\renewenvironment{abstract}{
 \vspace*{3cm}
 \begin{center}%
    \bfseries\abstractname
  \end{center}}%
  {\vfill}

\author{Marceli Grabowski}
\title{Wieloplatformowa aplikacja mobilna do zakupu biletów kinowych w technologii Xamarin}

\begin{document}

\includepdf[pages={1}]{titlepage.pdf}
\tableofcontents{}

\clearpage
\addcontentsline{toc}{chapter}{Streszczenie}
\abstract{Niniejsza praca opisuje projekt oraz implementację aplikacji mobilnej stworzonej za pomocą platformy Xamarin, na systemy Android oraz iOS, umożliwiającej zakup oraz rezerwację biletów kinowych. Pierwsza część pracy przedstawia opis technologii oraz narzędzi, wymagań dla aplikacji oraz architektury. W drugiej części pracy są zawarte szczegóły dotyczące projektu oraz wybranych aspektów fazy implementacji.}
\selectlanguage{english}
\abstract{This thesis describes project and implementation of mobile app created with Xamarin platform targeting Android and iOS systems, which allows to buy and reserve cinema tickets. First part of thesis contains descriptions of: technology, tools, application requirements and architecture. Second part of thesis contains details regarding project and selected aspects of implementation phase.}
\selectlanguage{polish}
\clearpage

\begin{flushright}
\vspace*{\fill}
\textit{Wszystkim tym, których życzliwość pozwoliła na powstanie tej pracy}
\end{flushright}


\pagebreak

\chapter*{Wstęp}
\chapter{Wprowadzenie do problematyki}
\chapter{Technologie i narzędzia wykorzystywane w pracy}
\chapter{Założenia projektowe}

\section{Przedmiot pracy}
Przedmiotem pracy jest utworzenie aplikacji mobilnej na platformy Android oraz iOS, umożliwiajacej rezerwację i zakup biletów kinowych w ramach sieci kin, wraz z towarzyszącą aplikacją serwerową. Aplikacja kliencka będzie utworzona w oparciu o platformę Xamarin, natomiast aplikacja serwerowa w oparciu o framework ASP.NET Core
\section{Wymagania funkcjonalne}
Oprogramowanie powinno spełniać następujące wymagania funkcjonalne:
\begin{enumerate}
\item Item
\end{enumerate}
\section{Wymagania niefunkcjonalne}
\section{Opis podstawowej architektury systemu}
\chapter{Projekt aplikacji}
\chapter{Implementacja}
\chapter{Podsumowanie}
\chapter*{Bibliografia}
\chapter*{Załączniki}
\end{document}