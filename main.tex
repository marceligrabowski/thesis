\chapter*{Wstęp}
\addcontentsline{toc}{chapter}{Wstęp}
\chapter{Wprowadzenie do problematyki}
\chapter{Technologie i narzędzia wykorzystywane w pracy}
\section{Xamarin.Android}
\section{Xamarin.iOS}
\section{ASP.NET Core}
\section{Entity Framework Core}
\section{Dapper}
\section{MVVMCross}
\section{AutoFac}
\section{Team Foundation Server}
\section{HockeyApp}
\section{Microsoft Azure}
\chapter{Założenia projektowe}
\section{Przedmiot pracy}
Przedmiotem pracy jest utworzenie aplikacji mobilnej na platformy Android oraz iOS, umożliwiającej rezerwację i zakup biletów kinowych w ramach sieci kin, wraz z towarzyszącą aplikacją serwerową. Aplikacja kliencka będzie utworzona w oparciu o platformę Xamarin, natomiast aplikacja serwerowa w oparciu o framework ASP.NET Core
\section{Wymagania funkcjonalne}
System powinien spełniać następujące wymagania funkcjonalne. Przy definiowaniu wymagań przyjęto następujących aktorów - Klient, Pracownik kina, Administrator systemu, System
\begin{enumerate}
\item Klient ma możliwość stworzenia konta użytkownika na podstawie adresu e-mail, w celu zachowania preferencji użytkownika pomiędzy urządzeniami.
\item Klient ma możliwość stworzenia konta użytkownika z wykorzystaniem konta w jednym ze wspieranych portali społecznościowych, w celu przyspieszenia procesu tworzenia konta.
\item Klient ma możliwość modyfikacji informacji o koncie użytkownika.
\item Klient ma możliwość zresetowania hasła do konta użytkownika, w celu odzyskania dostępu do konta. \newline
\item Klient ma możliwość wyboru domyślnego kina, w celu łatwiejszego dostępu do aktualnego repertuaru.
\item Klient ma możliwość przeglądania aktualnego repertuaru w danym kinie.
\item Klient ma możliwość przeglądania podstawowych informacji o filmie z repertuaru \newline
\item Klient ma możliwość złożenia rezerwacji biletu(ów) na wybrany seans w wybranym kinie.
\item Klient ma możliwość modyfikacji wybranej rezerwacji do 30 minut przed planowanym początkiem seansu.
\item System anuluje wszystkie niepotwierdzone rezerwacje 30 minut przed planowanym początkiem seansu. \newline
\item Klient ma możliwość zakupu biletu(ów) na wybrany seans w wybranym kinie.
\item Klient ma możliwość dokonania zapłaty za zakupione bilety za pośrednictwem zewnętrznego systemu płatności elektronicznych.
\item Klient ma możliwość zwrotu zakupionych biletów do 3 godzin przed planowanym seansem.
\item Klient ma możliwość wyboru miejsc na podstawie widoku sali kinowej.
\item Klient ma możliwość wyboru rodzaju biletu przy wyborze miejsc.
\item Klient ma możliwość przeglądania historii rezerwacji oraz zakupionych biletów \newline
\item Pracownik kina ma możliwość modyfikacji podstawowych danych o kinie.
\item Pracownik kina ma możliwość modyfikacji informacji o salach dostępnych w kinie.
\item Pracownik kina ma możliwość modyfikacji repertuaru kina.
\item Pracownik kina ma możliwość potwierdzenia rezerwacji klienta.
\item Pracownik kina ma możliwość anulowania rezerwacji klienta.
\item Pracownik kina ma możliwość dokonania sprzedaży biletów klientom w kasie biletowej.
\item Administrator systemu ma możliwość dodawania, modyfikowania i usuwania informacji o kinach należących do sieci kin.
\item Administrator systemu ma możliwość tworzenia oraz modyfikacji kont użytkowników oraz przydzielania im ról.
\item Użytkownik ma mieć możliwość stworzenia konta użytkownika na podstawie podanego adresu e-mail.
\item Użytkownik ma mieć możliwość stworzenia konta użytkownika na podstawie istniejącego konta w portalu społecznościowym.
\item Użytkownik ma mieć możliwość edycji danych konta użytkownika. (adres e-mail, hasło)
\item Użytkownik ma mieć możliwość zresetowania zapomnianego hasła do konta użytkownika.
\item Użytkownik ma mieć możliwość usunięcia konta użytkownika.
\item Użytkownik ma mieć możliwość dokonania rezerwacji biletu(ów) na wybrany seans filmowy w wybranym kinie.
\item Użytkownik ma mieć możliwość dokonania zakupu biletu(ów) na wybrany seans filmowy w wybranym kinie.
\item Użytkownik ma mieć możliwość zapłaty za zakupione bilety poprzez zewnętrzny serwis obsługujący płatności elektroniczne.
\item Użytkownik ma mieć możliwość przeglądania historii rezerwacji oraz historii zakupów.
\item Użytkownik ma mieć możliwość przeglądania repertuaru wybranego kina.
\item Użytkownik ma mieć możliwość przeglądania podstawowych informacji o wybranym filmie.
\item Użytkownik ma mieć możliwość modyfikacji rezerwacji do 30 minut przed seansem.
\item Użytkownik ma mieć możliwość anulowania rezerwacji do 30 minut przed seansem.
\item Niepotwierdzone rezerwacje są automatycznie anulowane 30 minut przed seansem.
\item Użytkownik ma mieć możliwość modyfikacji podstawowych danych dotyczących wybranego kina.
\item Użytkownik ma mieć możliwość modyfikacji repertuaru obowiązującego w wybranym kinie.
\item Użytkownik ma mieć możliwość podglądu podstawowych danych dotyczących wybranego kina.
\item Użytkownik ma mieć możliwość podglądu widoku sali kinowej w celu wybrania interesującego go miejsca.
\item Użytkownik ma mieć możliwość dokonania zakupu biletu(ów) bez potrzeby zakładania konta. (Informacje są przechowywane lokalnie na urządzeniu)
\item Użytkownik ma mieć możliwość dokonania rezerwacji biletu(ów) bez potrzeby zakładania konta. (Informacje są przechowywane lokalnie na urządzeniu)
\item Użytkownik ma mieć możliwość okazania biletu w formacie kodu QR.
\item Użytkownik ma mieć możliwość okazania biletu pomimo braku połączenia z siecią Internet.
\item Użytkownik ma mieć możliwość wybrania rodzaju biletu przy wybieraniu miejsca na sali kinowej.
\end{enumerate}
\section{Wymagania niefunkcjonalne}
Zbiór tych wymagań definiuje, jakie wymagania na system mają zostać spełnione, oprócz wymagań funkcjonalnych. Wymagania te głównie dotyczą wydajności, bezpieczeństwa i tym podobnych aspektów.
\begin{enumerate}
\item System powinien być dostępny w każdy dzień tygodnia, całą dobę.
\item System jest w stanie obsługiwać wiele jednocześnie podłączonych urządzeń.
\item Do poprawnego korzystania ze wszystkich funkcji oferowanych przez aplikację, wymagane jest stałe połączenie internetowe.
\item W celu zapewnienia odpowiedniego poziomu bezpieczeństwa, połączenie między serwerem i klientem ma być szyfrowane.
\item System ma wspierać również mechanizm sesji, jako dodatkowy mechanizm zabezpieczający połączenie.
\item Aplikacja kliencka powinna być dostępna na systemach Android (w wersji 4.4 i wyższej) oraz iOS (w wersji 8.0 i wyższej).
\item Aplikacja kliencka powinna zostać uruchomiona na urządzeniu mobilnym niezależnie od stanu połączenia internetowego.
\item Aplikacja serwerowa powinna móc być uruchomiona na serwerach z systemami rodziny Windows Server (wersja 2012 R2 i wyżej) oraz Linux

\end{enumerate}
\section{Opis podstawowej architektury systemu}
\chapter{Projekt aplikacji}
\section{Przypadki użycia}
\section{Interfejs}
\section{Diagram klas}
\addtocontents{toc}{\protect\newpage}
\chapter{Implementacja}
\section{DevOps}
\section{Autoryzacja użytkowników aplikacji}
\section{Synchronizacja danych offline-online}
\section{Bezpieczeństwo aplikacji}
\section{Implementacja wzorca CQRS}
\section{Testy interfejsu aplikacji}
\chapter{Podsumowanie}
\bibliographystyle{plain}
\bibliography{./bib/bib} 
\chapter*{Załączniki}
\addcontentsline{toc}{chapter}{Załączniki}
\section*{Spis tabel}
\addcontentsline{toc}{section}{Spis tabel}
\listoftables
\section*{Spis rysunków}
\addcontentsline{toc}{section}{Spis rysunków}
\listoffigures
\lstlistoflistings
\addcontentsline{toc}{section}{Spis listingów}
\section*{Instrukcja kompilacji i testowego uruchomienia aplikacji}
\addcontentsline{toc}{section}{Instrukcja kompilacji i testowego uruchomienia aplikacji}