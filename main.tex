\chapter*{Wstęp}
\chapter{Wprowadzenie do problematyki}
\chapter{Technologie i narzędzia wykorzystywane w pracy}
\section{Xamarin.Android}
\section{Xamarin.iOS}
\section{ASP.NET Core}
\section{Entity Framework Core}
\section{Dapper}
\section{MVVMCross}
\section{AutoFac}
\section{Team Foundation Server}
\section{HockeyApp}
\section{Microsoft Azure}
\chapter{Założenia projektowe}
\section{Przedmiot pracy}
Przedmiotem pracy jest utworzenie aplikacji mobilnej na platformy Android oraz iOS, umożliwiającej rezerwację i zakup biletów kinowych w ramach sieci kin, wraz z towarzyszącą aplikacją serwerową. Aplikacja kliencka będzie utworzona w oparciu o platformę Xamarin, natomiast aplikacja serwerowa w oparciu o framework ASP.NET Core
\section{Wymagania funkcjonalne}
Oprogramowanie powinno spełniać następujące wymagania funkcjonalne:
\begin{enumerate}
\item Item
\end{enumerate}
\section{Wymagania niefunkcjonalne}
Zbiór tych wymagań definiuje, jakie wymagania na system mają zostać spełnione, oprócz wymagań funkcjonalnych. Wymagania te głównie dotyczą wydajności, bezpieczeństwa i tym podobnych aspektów.
\begin{enumerate}
\item System powinien być dostępny w każdy dzień tygodnia, całą dobę.
\item System jest w stanie obsługiwać wiele jednocześnie podłączonych urządzeń.
\item Do poprawnego korzystania ze wszystkich funkcji oferowanych przez aplikację, wymagane jest stałe połączenie internetowe.
\item W celu zapewnienia odpowiedniego poziomu bezpieczeństwa, połączenie między serwerem i klientem ma być szyfrowane.
\item System ma wspierać również mechanizm sesji, jako dodatkowy mechanizm zabezpieczający połączenie.
\item Aplikacja kliencka powinna być dostępna na systemach Android (w wersji 4.4 i wyższej) oraz iOS (w wersji 8.0 i wyższej).
\item Aplikacja kliencka powinna zostać uruchomiona na urządzeniu mobilnym niezależnie od stanu połączenia internetowego.
\item Aplikacja serwerowa powinna móc być uruchomiona na serwerach z systemami rodziny Windows Server (wersja 2012 R2 i wyżej) oraz Linux

\end{enumerate}
\section{Opis podstawowej architektury systemu}
\chapter{Projekt aplikacji}
\section{Przypadki użycia}
\section{Interfejs}
\section{Diagram klas}
\addtocontents{toc}{\protect\newpage}
\chapter{Implementacja}
\section{DevOps}
\section{Autoryzacja użytkowników aplikacji}
\section{Synchronizacja danych offline-online}
\section{Bezpieczeństwo aplikacji}
\section{Implementacja wzorca CQRS}
\section{Testy interfejsu aplikacji}
\chapter{Podsumowanie}
\chapter*{Bibliografia}
\addcontentsline{toc}{chapter}{Bibliografia}
\chapter*{Załączniki}
\addcontentsline{toc}{chapter}{Załączniki}
\section*{Spis tabel}
\addcontentsline{toc}{section}{Spis tabel}
\section*{Spis rysunków}
\addcontentsline{toc}{section}{Spis rysunków}
\section*{Spis listingów}
\addcontentsline{toc}{section}{Spis listingów}
\section*{Instrukcja kompilacji i testowego uruchomienia aplikacji}
\addcontentsline{toc}{section}{Instrukcja kompilacji i testowego uruchomienia aplikacji}